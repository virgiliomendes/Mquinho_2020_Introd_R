 %\documentclass[a4paper, oneside, 12pt]{article}
 
 \usepackage{mathtools}
 \usepackage[brazil]{babel}
 \usepackage{indentfirst}
 \usepackage{ragged2e}
 \usepackage{enumitem}

\usepackage{hyperref}
\hypersetup{pdftitle={\@title},pdfauthor={\@author}, breaklinks=true,linktoc=page, bookmarksopen=false,bookmarksnumbered=true, pdfstartview={XYZ null null 1}, bookmarksdepth=4,
  	colorlinks=true, linkcolor=main, filecolor=main, urlcolor=main, citecolor=main,}

\usepackage[htt]{hyphenat} %To break long \texttt text
\usepackage[bottom]{footmisc} %Para fixar as notas de rodapé no final da página.  
\usepackage{footnotebackref}
  
\usepackage{xcolor}
\definecolor{main}{rgb}{0.0, 0.5, 1.0}

\usepackage{geometry}
\geometry{top=2cm, bottom=2cm, left=2cm, right=2cm, includefoot}

\setcounter{secnumdepth}{2} %Numerar seções e subseções (1 se somente seções)

\usepackage{titlesec}
\titlespacing*{\section}
{0pt}{10pt}{10pt}
\titlespacing*{\subsection}
{0pt}{10pt}{10pt}
\titlespacing*{\subsubsection}
{0pt}{10pt}{10pt}

\titleformat*{\section}{\large \bfseries}
\titleformat*{\subsection}{\normalsize \bfseries}
\titleformat*{\subsubsection}{\setmainfont{TeX Gyre Pagella}\scshape\bfseries}

\usepackage{setspace}

\newcommand{\titulo}[1]{\footnotesize MQuinho 2020 \\
                    \huge Introdução ao R \\
                    \textbf{#1}}

\newcommand{\autor}[2]{\Large #1\thanks{E-mail: #2}}

\AtEndDocument{\nocite{*}}  
